

\section{Overview}
The livestock industry has played a vital role in world’s economy. In colonial times animals were allowed to roam and hunt in the forest. This industry was dependent on free grazing. The livestock industry began to transform after the Civil War (1917 – 1922). This was the time when the advancement in technology began.  The technology started to change agriculture , free range evolved into a more stable farming environment and different agriculture state, societies were formed [1]. 


The growth rate of livestock production is significantly more than the agriculture production in most developing countries. According to experts, this trend is likely to continue over the next 20 years and estimated at 4.5 percent per annum. In livestock business, to produce more intensive products, the feed is one of the major cost components. Therefore, It is needed to be carefully assessed with estimated feed requirements in mind [2]. The feed can be optimized based on the weight of the animal.

\section{Problem Statement}

The profit in livestock business depends on the weight and body size of the cattle. Weight is the most important factor in Livestock business. It is very effective in assessing the reproductive efficiency and growth performance of an animal. Weight is also used in measuring the correct dose of therapeutic pharmaceutical to treat diseases that affect cattle. The correct amount of feed can also be determined based on the weight to avoid underfeeding or overfeeding [3]. But Keeping track of the weight on a daily basis is the most tedious task and requires a lot of manhandling and is time consuming. 
\section{Proposed Solution}

According to the work done by Chintan Bhatt et. al [4], image processing and the algorithms of machine learning can be used to estimate the weight of the cattle remotely without even touching them. It has been observed in research study that by implementing this method 73 - 89\% accuracy can be achieved [4]. But most of the work is dependent on additional cameras or hardware to achieve the goal.

In our project we want to achieve this goal only using a mobile phone’s camera. And aim to develop a system which will be able to keep track of daily weight of Cattles.  
In our project, image processing, image segmentation and machine learning algorithms would be used to achieve the high accuracy for the task of estimating the weights of the cattle [5].  There are many ways to implement these techniques to get the task done but all those methods have their own advantages as well as limitations. Also, there is a difference of accuracy in between them. We will implement the algorithms in a way that high accuracy can be achieved.  

Over the period of one year, we will be researching and developing a model, which will be able to predict animals weight. 

Our project has three main components, the core and web client.
\subsection{{Core Component:}}
The core component will be responsible to detect the features of the animal from the image to estimate/predict the weight of the animal. This core component will be developed using machine learning and image processing algorithms. 
\subsection{Web Client:}
Web client will consist of admin panel. The admin panel would display the overall growth of cattle, in a last past years. It will show the overall growth as well as growth of animals at individual level. All basic information of each animal can be seen by admin panel. 


\section{Intended User}

Our InstaWeight is a special purpose system and therefore will be used only by ranchers and cattle farmers. The mobile application would be used by cattle farmer, while the admin panel would be used by the cattle farm's owner who will keep track of his animal growth to run his business. 

\section{Key Challenges}
This section mentions the key challenges that we foresee in this project and possible ways to address them.

\subsection{Availibilty of Cattle's Dataset and Weight Labels}
The biggest challenge in our project is the availability of data related to our problem. Although we were able to collect RGB images of cattle by visiting some farms but we were unable to get weight labels. and after covid 19 our data collection process was impacted badly.




